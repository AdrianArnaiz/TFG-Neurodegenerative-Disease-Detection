\apendice{Documentación de usuario}

\section{Introducción}
En esta sección explicaremos lo necesario para que lo usuarios puedan instalar y utilizar la aplicación.

\section{Requisitos de usuarios}
El usuario, para poder utilizar la aplicación, deberá tener instalado Python, al menos la version 3.6.8, e instalar los requerimientos estipulados en  \texttt{src/demo/requirements.txt}. Idealmente, el sistema se instalará en Windows 10, ya que la aplicación está optimizad para este sistema operativo. Sin embargo, funciona en todos los sistemas operativos probados (Windows, Linux y Mac).

\section{Instalación}
Los pasos para la instalación del proyecto se detallan en \ref{subsec:instalar}. Les resumimos:
\begin{enumerate}
\item Clonar repositorio e ir al directorio \texttt{src/demo} o , en caso de tener el \textit{release}, abrir únicamente el \textit{release} en el directorio raíz.
\item \item Descargar \myurl{https://mega.nz/\#!fRRFSKrT!0EMBqtYjogQrSQgudWifOAXm\_A5Yx9UnX5Qk\_Enanuk}{archivo de pesos del modelo} \textit{VGGish} y alojarlo en la carpeta \texttt{prediccion}.
\item Ejecutar install.cmd o pip install -r requirements.txt
\end{enumerate}

Para ejecutar la aplicación:
\begin{center}
python main.py
\end{center}

\section{Manual del usuario}
Detallaremos como realizar las operaciones principales de la aplicación.

\subsection{Cargar y borrar un audio}

\subsubsection{Cargar un audio}
Para cargar un audio, ver Figura \ref{fig:cargaaudio}:
\begin{enumerate}
\item Clickar botón `+ Añadir'
\item Elegir audio o audios tipo wav en el menú desplegable.
\item Los audios aparecerán listados.
\end{enumerate}

\imagen{cargaaudio}{Carga de audios en la aplicación}

\subsubsection{Borrar un audio}
Para borrar un audio, ver Figura \ref{fig:borraaudio}:
\begin{enumerate}
\item Elegir un audio. Destacamos que cuando digamos elegir audio de aquí en adelante, nos referimos a clickar encima del audio. Es decir clickar encima del nombre del audio en la lista de audios cargados, y este se pondrá remarcado en azul. Cuando esté remarcado en azul, consideraremos que el audio está elegido.
\item Clickar botón `Borrar'.
\end{enumerate}

\imagen{borraaudio}{Borra audios de la aplicación}


\subsection{Reproducir un audio}
Para reproducir un audio, ver Figura \ref{fig:reproduceaudio}:
\begin{enumerate}
\item Elegir un audio. 
\item Clickar botones de control de reproducción: \textit{play}, \textit{pause}, \textit{stop}, \textit{rewind}, \textit{mute} o \textit{volume}.
\item El audio se reproducirá y se detallará en pantalla un reloj de reproducción.
\end{enumerate}

\imagen{reproduceaudio}{Reproduce un audio de la aplicación}

\subsection{Predice un audio y vea gráficas}
Para predecir un audio un audio, ver Figura \ref{fig:prediceaudio}:
\begin{enumerate}
\item Elegir un audio. 
\item Elegir el sexo.
\item Introducir edad en formato entero mayor que 0.
\item Clickar botón `Predecir audio'.
\item Se mostrará en pantalla el resultado de la predicción en texto. Se mostrará si el paciente está sano o no y con que probabilidad (color de letra verde si sano, rojo si PD). Se mostrará en pantalla las dos gráficas de análisis del audio: amplitud y espectrograma de frecuencias.
\end{enumerate}

\imagen{prediceaudio}{Predice un audio de la aplicación}