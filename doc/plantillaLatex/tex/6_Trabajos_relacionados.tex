\capitulo{6}{Trabajos relacionados}

En este apartado se comentará el estado del arte de la materia y algunos trabajos relacionados.\newline

Primero trataremos los trabajos de la base da datos adquirida. Estos trabajos corresponden a Giovanni Dimauro de la Universidad de Bari.
\begin{itemize}
	\item \textit{VoxTester, software for digital evaluation of speech changes in Parkinson disease} \cite{giovanni1}.\\
	Este artículo describe la realización de una herramienta software para la ayuda en la evaluación de los cambios y variaciones en la voz de pacientes con la enfermedad del Parkinson. El funcionamiento de la herramienta es el siguiente: insertamos una serie de audios del paciente, la herramienta analiza el audio y nos devuelve 4 gráficas que un experto deberá interpretar para la evaluación de la enfermedad. Las gráficas de salida son las siguientes: gráfica del DDK rate, gráfica de la intensidad vocal y duración del discurso, gráfica del espectro de frecuencias vocal y gráfica del nivel de presión vocal. Éstos parámetros sacados de la voz son importantes ya que pueden ser indicadores de la enfermedad del Parkinson. Por ejemplo, un rango de frecuencias de la voz pequeño, una frecuencia fundamental baja o un decaimiento notable de la intensidad de la voz en el discurso serán una de las características del discurso para personas con esta enfermedad.\\
	En este experimento no se utiliza en ningún momento técnicas de clasificación ni tampoco la herramienta discierne por ella misma si la persona del audio tiene una enfermedad o en qué grado. Lo único que hace es analizar la onda sonora y mostrar algunas características, por lo que no se alinea del todo con nuestros objetivos.
	\item \textit{Assessment of Speech Intelligibility in Parkinson’s Disease Using a Speech-To-Text System} \cite{giovanni2}.\\
	Este experimento trata de resolver el problema del anterior (valuación de los cambios y variaciones en la voz de pacientes con la enfermedad del Parkinson) con otra línea. Para ello utilizará sistemas de reconocimiento de voz y transcripción a texto (Speech-To-Text Systems) para evaluar la inteligibilidad del discurso de las personas con Parkinson. El proceso será el siguiente: la persona con Parkinson habla a un sistema STT, el texto de la transcripción se pasa a un programa que evalúa su acierto y finalmente ese programa devuelve el porcentaje de fallo en el reconocimiento de voz respecto a la frase objetivo. El objetivo de este proyecto ha sido comparar los fallos de reconocimiento en las palabras entre 3 grupos de personas: jóvenes sanos, mayores sanos y pacientes de la enfermedad del Parkinson. Los resultados obtenidos es que había mucho mas fallo en el reconocimiento de palabras en los pacientes de la enfermedad del Parkinson.
	
\end{itemize}