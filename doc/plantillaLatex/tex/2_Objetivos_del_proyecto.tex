\capitulo{2}{Objetivos del proyecto}

\section{Objetivos generales}
\begin{itemize}
	\item Investigar y condensar el estado del arte de la investigación sobre la detección del Parkison a través de la voz resumiendo los artículos científicos más importantes, identificando las tecnologías, herramientas y procesos actuales utilizadas en ellos, realizando taxonomías, etc.
	\item Recopilación de bases de datos adecuadas para la investigación, tanto para uso en este proyecto como para su uso en proyectos posteriores, cerciorando que son \english{datasets} de audios correctos para las tareas necesitadas (i.e. audios etiquetados).
	\item Realización de un estudio comparativo en cuanto a la utilización de diferentes modelos de clasificación y conjuntos de características extraídas de los audios. Se compararán resultados, tanto entre los diferentes experimentos que nosotros realicemos, como entre nuestros mejores experimentos y resultados científicos de anteriores experimentos (publicados en artículos científicos).
	\item Aportación de una nueva perspectiva a este campo de investigación: extracción de las características de los audios mediante \english{Deep Learning}.
	\item Finalmente, utilizar todo lo descrito anteriormente para realizar una aplicación la cual permita la monitorización de la diagnosis de la enfermedad del Parkinson a través de la voz. La aplicación será capaz de discernir, mediante un clasificador, si la persona de un audio subido a la aplicación web tiene Parkinson o no. Esta aplicación será un ejemplo de como poder plasmar los resultados de esta investigación en una herramienta funcional.
\end{itemize}

\section{Objetivos técnicos}
\begin{itemize}
	\item Desarrollar un algoritmo, cuya implementación en \textit{Python} permita la extracción de características de los audios de manera adecuada (envolver la extracción de características que realizan \textit{Disvoice} y \textit{VGGish} en clases).
	\item Desarrollar una aplicación de escritorio para la etapa de explotación, que muestre como se pueden plasmar los resultados de la investigación para en una herramienta funcional. 
	\item Utilizar las herramientas más adecuadas para la realización del estudio comparativo: valoraremos \textit{Python} y sus bibliotecas, \textit{Weka}...
	\item Utilizar un sistema de control de versiones Git utilizando para ello la plataforma \textit{Github}. Utilizar un cliente \textit{Git} para el trabajo local para lo cual utilizaremos TortoiseGit.
	\item Utilizar la metodología \textit{SCRUM} en la elaboración del proyecto y concretamente la herramienta \textit{ZenHub} (como extensión de \textit{Github}) para la ayuda en la gestión de proyectos.
	
	
\end{itemize}