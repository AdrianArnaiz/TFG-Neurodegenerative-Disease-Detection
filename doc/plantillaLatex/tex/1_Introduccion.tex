\capitulo{1}{Introducción}

\section{Introducción al proyecto}
En enfermedades como la depresión y las neurodegenerativas es requerido un seguimiento cercano de la enfermedad para ver su evolución. Actualmente es necesario que un facultativo médico se desplace a los domicilios de los pacientes y realice una serie de test y entrevistas estructuradas para evaluar la progresión de la enfermedad (depresión, alzheimer, \textbf{Parkinson}, etc). En líneas generales se pretende que el sistema propuesto sea capaz de determinar esa diagnosis solamente a partir de grabaciones de voz. En la actualidad, la monitorización de los síntomas y la diagnosis de enfermedades son costosos y, logísticamente, son inconveniente para el paciente y el personal clínico, lo que también dificulta el reclutamiento para futuros ensayos clínicos a gran escala. Por ello, todo lo que signifique reducir complejidad y costes en este ámbito ayudará tanto a pacientes como a profesionales.

El deterioro del habla del Parkinson se ha estudiado utilizando grabaciones de voz o conjuntos de datos de voz. Los conjuntos de datos de voz son características extraídas de las grabaciones de voz. Como en el análisis de voz patológica, la lectura ``The Rainbow Passage" también se usa en estudios de PD (Parkinson Detection). Sin embargo, las grabaciones de vocales sostenidas (pronunciación continua de una vocal durante un número determinado de segundos) son cada vez más populares como la entrada analítica. Holmes \cite{j2000voice}, estudió las características de la voz para etapas tempranas y posteriores de PD en comparación con personas sanas usando 4 segundos de vocal sostenida 'a' y grabaciones de monólogos de 1 minuto. 
Holmes utilizó las grabaciones para extraer ocho características para el análisis acústico a partir de la mitad de la vocal sostenida y el canto a escala. Estas características incluyen la frecuencia fundamental y su desviación estándar, la intensidad media y su desviación estándar, el rango de frecuencia de frecuencia máxima, la fluctuación de fase, el brillo y la relación de ruido a armónicos. Por ejemplo, en este estudio Holmes descubrió, entre otros muchos hallazgos, que en comparación con los controles y los datos normativos publicados previamente, las voces de los pacientes con PD \footnote{Parkinson Disease} en etapa temprana y tardía se caracterizaron perceptualmente por la variabilidad de tono y volumen limitados, la transpirabilidad, la dureza y el volumen reducido.

Desde entonces, un número cada vez mayor de investigadores han estado utilizando la vocal sostenida 'a' para estudiar la disfonía de la enfermedad del Parkinson, incluidos los investigadores Little y Tsanas 2010. En esta investigación realizada por Little y Tsanas en el 2010 \cite{MxLtAccurate}, demostraron la replicación rápida y remota de la evaluación UPDRS (escala de calificación de la enfermedad de Parkinson, véase sección \ref{sec:Updrs}) con una precisión clínicamente útil (aproximadamente 7.5 puntos UPDRS diferencia de las estimaciones de los clínicos), utilizando solo pruebas de voz simples, autoadministradas y no invasivas.

Por lo tanto, nuestro proyecto consistirá en lo siguiente. Debemos conseguir una base de datos de audios. Estos audios se deberán realizar las entrevistas estructuradas, en las cuales sus respuestas serán grabadas en el mismo momento de la realización. Las grabaciones se identificarán con el identificador del paciente, así como con datos adicionales, como edad y sexo, del paciente. Las grabaciones incluirán grabaciones estandar, donde siempre se recite la misma frase, la misma palabra o la misma vocal. Concretamente, la base de datos con la que trabajaremos se describe en  \cite{OrzCorpus}. Los audios de las grabaciones se procesarán para obtener diferentes atributos y características de la voz, como por ejemplo los usados en el estudio de Orozc et al. \cite{Orz2016}, o los anteriormente comentados que se usaron en el estudio de Holmes. En nuestro caso se obtendrán los indicadores tono, cadencia, volumen y ritmo para su posterior análisis. A partir de ahí se construirá y entrenará una serie de modelos para predecir si la persona de un audio tiene Parkinson o no. Se realizarán una serie de diferentes experimentos con el objetivo de crear el mejor modelo posible. Este modelo correctamente construido se utilizará para evaluar la enfermedad de un paciente. Con ello conseguiremos avanzar en las investigaciones relativas a las enfermedades neurodegenerativas y su diagnóstico, y también realizar un producto, aplicación, que ayude e intente mejorar la calidad de vida tanto a pacientes como a trabajadores sanitarios.




