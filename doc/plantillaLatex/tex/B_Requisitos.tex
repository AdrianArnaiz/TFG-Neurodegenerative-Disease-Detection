\apendice{Especificación de Requisitos}

\section{Introducción}
En esta sección se abordarán los diferentes objetivos y requisitos del proyecto. Se presentarán tanto los requisitos globales del proyecto, como los requisitos funcionales y casos de uso de la aplicación.

\section{Objetivos generales}
\begin{itemize}
	\item Investigar y condensar el estado del arte de la investigación sobre la detección del Parkison a través de la voz resumiendo los artículos científicos más importantes, identificando las tecnologías, herramientas y procesos actuales utilizadas en ellos, realizando taxonomías, etc.
	\item Recopilación de bases de datos adecuadas para la investigación, tanto para uso en este proyecto como para su uso en proyectos posteriores, cerciorando que son \english{datasets} de audios correctos para las tareas necesitadas (i.e. audios etiquetados).
	\item Realización de un estudio comparativo en cuanto a la utilización de diferentes modelos de clasificación y conjuntos de características extraídas de los audios. Se compararán resultados, tanto entre los diferentes experimentos que nosotros realicemos, como entre nuestros mejores experimentos y resultados científicos de anteriores experimentos (publicados en artículos científicos).
	\item Aportación de una nueva perspectiva a este campo de investigación: extracción de las características de los audios mediante \english{Deep Learning}.
	\item Finalmente, utilizar todo lo descrito anteriormente para realizar una aplicación la cual permita la monitorización de la diagnosis de la enfermedad del Parkinson a través de la voz. La aplicación será capaz de discernir, mediante un clasificador, si la persona de un audio subido a la aplicación web tiene Parkinson o no. Esta aplicación será un ejemplo de como poder plasmar los resultados de esta investigación en una herramienta funcional.
\end{itemize}

\section{Catalogo de requisitos}
En esta sección se enumerarán los requisitos funcionales de nuestra aplicación y los actores que la realizan.

Actores:
\begin{description}
\item[Usuario] El usuario será quien utiliza la aplicación. Idealmente, será el facultativo médico en su consulta.
\item[Científico de datos] Este actor será quien realiza la investigación, quien realiza los experimentos con los clasificadores.
\end{description}



\begin{itemize}
	\item \textbf{RF.1} Crear una herramienta que nos permita detectar con que probabilidad tiene Parkinson la persona de un audio dado.
	\begin{itemize}
	\item \textbf{RF.1.1} El usuario podrá elegir el audio que  quiere predecir.
	\item \textbf{RF.1.2} El usuario podrá añadir la edad y el sexo del paciente.
	\item \textbf{RF.1.3} El usuario podrá ver la probabilidad en porcentaje de PArkinson de la persona del audio.
	\end{itemize}
	
	\item \textbf{RF.2} Crear una herramienta que permita ver las gráficas de amplitud de onda y el espectrograma de frecuencias de un audio.
	\begin{itemize}
	\item \textbf{RF.2.1} El usuario podrá elegir el audio del que quiere ver las gráficas.
	\item \textbf{RF.2.2} El usuario podrá ver las gráficas en la pantalla.
	\item \textbf{RF.2.3} El usuario podrá navegar en las gráficas (zoom, moverse).
	\item \textbf{RF.2.4} El usuario podrá guardar las gráficas.
	\end{itemize}
	
	\item \textbf{RF.3} Crear una herramienta que permita reproducir un audio.
	\begin{itemize}
	\item \textbf{RF.3.1} El usuario podrá elegir el audio que quiere reproducir.
	\item \textbf{RF.3.2} El usuario podrá reproducir un audio dando al play.
	\item \textbf{RF.3.3} El usuario podrá pausar el audio dando al pause.
	\item \textbf{RF.3.4} El usuario podrá rebobinar el audio dando a rebobinar.
	\item \textbf{RF.3.5} El usuario podrá cambiar el volumen deslizando una pestaña.
	\item \textbf{RF.3.6} El usuario podrá enmudecer el audio pulsando al mute.
	\end{itemize}
	
	\item \textbf{RF.4} Crear una herramienta que permita cargar y borrar audios al entorno para trabajar con ellos.
	\begin{itemize}
	\item \textbf{RF.4.1} El usuario podrá elegir los audios de cualquier ruta que quiere cargar en la aplicación.
	\item \textbf{RF.4.2} El usuario podrá elegir los audios cargados en la aplicación que quiere borrar.
	\end{itemize}
	
	\item \textbf{RF.5} Crear modelos de clasificación para los audios con Python.
	\begin{itemize}
	\item \textbf{RF.5.1} El científico de datos podrá cargar datos de los audios.
	\item \textbf{RF.5.2} El científico de datos podrá elegir el tipo de modelo que queremos entrenar.
	\item \textbf{RF.5.3} El científico de datos podrá elegir la validación cruzada que queremos realizar.
	\item \textbf{RF.5.4} El científico de datos podrá particionar los datos de manera estratificada.
	\item \textbf{RF.5.5} El científico de datos podrá entrenar el modelo con la anterior configuración.
	\end{itemize}
	
	\item \textbf{RF.6} Evaluar clasificadores.
	\begin{itemize}
	\item \textbf{RF.6.1} El científico de datos podrá obtener medidas de rendimiento de los clasificadores.
	\item \textbf{RF.6.2} El científico de datos podrá resumir medidas de rendimiento de todos los clasificadores.
	\item \textbf{RF.6.3} El científico de datos podrá mostrar medidas de rendimiento en tablas o gráficas para ayudar a la decisión.
	\end{itemize}
\end{itemize}

\section{Especificación de requisitos}
\subsection{Diagrama de casos de uso}

\imagen{DiagCasosUso}{Diagrama de casos de uso de la aplicación.}

\imagen{DiagCasosUso2}{Diagrama de casos de uso de la investigación.}

\subsection{Especificación de los casos de uso}

%CU0
\tablaSmallSinColores{Caso de uso 0: Insertar audios.}{p{3cm} p{.75cm} p{9.5cm}}{tablaUC0}{
  \multicolumn{3}{l}{Caso de uso 0: Insertar audios.} \\
 }
 {
  Descripción                            & \multicolumn{2}{p{10.25cm}}{Permite al usuario cargar audios de su equipo a la aplicación.} \\\hline
  \multirow{2}{3.5cm}{Requisitos}   &\multicolumn{2}{p{10.25cm}}{RF-4} \\\cline{2-3}
                                         & \multicolumn{2}{p{10.25cm}}{RF-4.1} \\\cline{2-3}
                                         & \multicolumn{2}{p{10.25cm}}{RF-4.2}
                                         \\\hline
  Precondiciones                         &  \multicolumn{2}{p{10.25cm}}{Ninguna}   \\\hline
  \multirow{2}{3.5cm}{Secuencia normal}  & Paso & Acción \\\cline{2-3}
                                         & 1    & El usuario pincha el botón 'Añadir'.
  \\\cline{2-3}
                                         & 2    & Se despliega un menú de búsqueda de archivos.
  \\\cline{2-3}
                                         & 3    & Se eligen los audios y se da a 'Abrir'
    \\\cline{2-3}
                                         & 4    & Se cargan las imágenes dentro de la aplicación.
                                         \\\hline
  Postcondiciones                        & \multicolumn{2}{p{10.25cm}}{Los audios aparecen en el recuadro de audios cargados en la aplicación, listos para reproducir, predecir o mostrar gráficas.} \\\hline
  Excepciones                        & \multicolumn{2}{p{10.25cm}}{Audio no encontrado}\\\hline
  Importancia                            & Alta \\\hline
  Urgencia                               & Alta \\
}

%CU1
\tablaSmallSinColores{Caso de uso 1: Predecir Audio.}{p{3cm} p{.75cm} p{9.5cm}}{tablaUC1}{
  \multicolumn{3}{l}{Caso de uso 1: Predecir Audio.} \\
 }
 {
  Descripción                            & \multicolumn{2}{p{10.25cm}}{Permite al usuario obtener la predicción de parkinson de un audio cargado en la aplicación.} \\\hline
  \multirow{2}{3.5cm}{Requisitos}   &\multicolumn{2}{p{10.25cm}}{RF-1} \\\cline{2-3}
                                         & \multicolumn{2}{p{10.25cm}}{RF-1.1} \\\cline{2-3}
                                         & \multicolumn{2}{p{10.25cm}}{RF-1.2} \\\cline{2-3}
                                         & \multicolumn{2}{p{10.25cm}}{RF-1.3}
                                         \\\hline
  Precondiciones                         &  \multicolumn{2}{p{10.25cm}}{Tener un audio cargado}  \\\hline
  \multirow{2}{3.5cm}{Secuencia normal}  & Paso & Acción \\\cline{2-3}
                                         & 1    & El usuario elige un audio de la lista.
  \\\cline{2-3}
                                         & 2    & El usuario introduce edad y sexo del paciente.
  \\\cline{2-3}
                                         & 3    & El usuario pincha el botón predecir.
    \\\cline{2-3}
                                         & 4    & Aparece la predicción en pantalla.
                                         \\\hline
  Postcondiciones                        & \multicolumn{2}{p{10.25cm}}{Aparece la predicción del audio en porcentaje en la pantalla.} \\\hline
  Excepciones                        & \multicolumn{2}{p{10.25cm}}{Audio no seleccionado. Edad no introducida. Edad no está en formato válido.}\\\hline
  Importancia                            & Alta \\\hline
  Urgencia                               & Alta \\
}

%CU2
\tablaSmallSinColores{Caso de uso 2: Ver Gráficas.}{p{3cm} p{.75cm} p{9.5cm}}{tablaUC2}{
  \multicolumn{3}{l}{Caso de uso 2: Ver gráficas.} \\
 }
 {
  Descripción                            & \multicolumn{2}{p{10.25cm}}{Permite al usuario obtener las gráficas de un audio.} \\\hline
  \multirow{2}{3.5cm}{Requisitos}   &\multicolumn{2}{p{10.25cm}}{RF-2} \\\cline{2-3}
                                         & \multicolumn{2}{p{10.25cm}}{RF-2.1} \\\cline{2-3}
                                         & \multicolumn{2}{p{10.25cm}}{RF-2.2} \\\cline{2-3}
                                         & \multicolumn{2}{p{10.25cm}}{RF-2.3} \\\cline{2-3}
                                         & \multicolumn{2}{p{10.25cm}}{RF-2.4}
                                         \\\hline
  Precondiciones                         &  \multicolumn{2}{p{10.25cm}}{Tener un audio cargado y con los detalles del paciente introducidos.}  \\\hline
  \multirow{2}{3.5cm}{Secuencia normal}  & Paso & Acción \\\cline{2-3}
                                         & 1    & El usuario elige un audio de la lista.
  \\\cline{2-3}
                                         & 2    & El usuario pincha el botón predecir.
  \\\cline{2-3}
                                         & 3    & Aparecen las gráficas en pantalla.
    \\\cline{2-3}
                                         & 4    & El usuario puede navegar o guardar las gráficas.
                                         \\\hline
  Postcondiciones                        & \multicolumn{2}{p{10.25cm}}{Aparece las gráficas de amplitud y espectrograma de frecuencias en la pantalla.} \\\hline
  Excepciones                        & \multicolumn{2}{p{10.25cm}}{Audio no seleccionado.}\\\hline
  Importancia                            & Alta \\\hline
  Urgencia                               & Alta \\
}

%CU3
\tablaSmallSinColores{Caso de uso 3: Reproducir audios.}{p{3cm} p{.75cm} p{9.5cm}}{tablaUC3}{
  \multicolumn{3}{l}{Caso de uso 3: Reproducir audios.} \\
 }
 {
  Descripción                            & \multicolumn{2}{p{10.25cm}}{Permite al usuario reproducir un audio.} \\\hline
  \multirow{2}{3.5cm}{Requisitos}   &\multicolumn{2}{p{10.25cm}}{RF-3} \\\cline{2-3}
                                         & \multicolumn{2}{p{10.25cm}}{RF-3.1} \\\cline{2-3}
                                         & \multicolumn{2}{p{10.25cm}}{RF-3.2} \\\cline{2-3}
                                         & \multicolumn{2}{p{10.25cm}}{RF-3.3} \\\cline{2-3}
                                         & \multicolumn{2}{p{10.25cm}}{RF-3.4} \\\cline{2-3}
                                         & \multicolumn{2}{p{10.25cm}}{RF-3.5} \\\cline{2-3}
                                         & \multicolumn{2}{p{10.25cm}}{RF-3.6}
                                         \\\hline
  Precondiciones                         &  \multicolumn{2}{p{10.25cm}}{Tener un audio cargado}  \\\hline
  \multirow{2}{3.5cm}{Secuencia normal}  & Paso & Acción \\\cline{2-3}
                                         & 1    & El usuario elige un audio de la lista.
  \\\cline{2-3}
                                         & 2    & El usuario pincha el botón play.
  \\\cline{2-3}
                                         & 3    & El sonido comienza a sonar.
    \\\cline{2-3}
                                         & 4    & El usuario puede pinchar el botón pause, volumen o rebobinar.
   \\\cline{2-3}
                                         & 5    & El audio se para, cambia de volumen o vuelve a empezar.
                                         \\\hline
  Postcondiciones                        & \multicolumn{2}{p{10.25cm}}{Ninguna} \\\hline
  Excepciones                        & \multicolumn{2}{p{10.25cm}}{Audio no seleccionado.}\\\hline
  Importancia                            & Alta \\\hline
  Urgencia                               & Alta \\
}

%CU4
\tablaSmallSinColores{Caso de uso 4: Introducir detalles del paciente}{p{3cm} p{.75cm} p{9.5cm}}{tablaUC4}{
  \multicolumn{3}{l}{Caso de uso 4: Introducir detalles del paciente} \\
 }
 {
  Descripción                            & \multicolumn{2}{p{10.25cm}}{Permite al usuario introducir la edad y el sexo del paciente para su posterior predicción.} \\\hline
  \multirow{2}{3.5cm}{Requisitos}   & \multicolumn{2}{p{10.25cm}}{RF-1} \\\cline{2-3}
  										&\multicolumn{2}{p{10.25cm}}{RF-1.2} 
                                         \\\hline
  Precondiciones                         &  \multicolumn{2}{p{10.25cm}}{Tener un audio cargado y seleccionado}  \\\hline
  \multirow{2}{3.5cm}{Secuencia normal}  & Paso & Acción \\\cline{2-3}
                                         & 1    & El usuario pincha en hombre o mujer.
  \\\cline{2-3}
                                         & 2    & El usuario introduce la edad (entero mayor que 0).
  \\\cline{2-3}
                                         & 3    & El usuario puede dar a predecir.
                                         \\\hline
  Postcondiciones                        & \multicolumn{2}{p{10.25cm}}{La predicción se realiza de manera satisfactoria ya que los datos del paciente del audio se han leído correctamente.} \\\hline
  Excepciones                        & \multicolumn{2}{p{10.25cm}}{Edad no insertada. Edad en formato erróneo.}\\\hline
  Importancia                            & Alta \\\hline
  Urgencia                               & Alta \\
}

%CU5
\tablaSmallSinColores{Caso de uso 5: Ver predicción del audio}{p{3cm} p{.75cm} p{9.5cm}}{tablaUC5}{
  \multicolumn{3}{l}{Caso de uso 5: Ver predicción del audio} \\
 }
 {
  Descripción                            & \multicolumn{2}{p{10.25cm}}{Permite al usuario ver el porcentaje de parkinson con el color indicado.} \\\hline
  \multirow{2}{3.5cm}{Requisitos}   & \multicolumn{2}{p{10.25cm}}{RF-1} \\\cline{2-3}
  									&\multicolumn{2}{p{10.25cm}}{RF-1.3} 
                                         \\\hline
  Precondiciones                         &  \multicolumn{2}{p{10.25cm}}{Tener un audio cargado, seleccionado y con los datos del paciente introducidos}  \\\hline
  \multirow{2}{3.5cm}{Secuencia normal}  & Paso & Acción \\\cline{2-3}
                                         & 1    & El usuario pincha en predicción.
  \\\cline{2-3}
                                         & 2    & Se clasifica el audio.
  \\\cline{2-3}
                                         & 3    & Se muestra la predicción del audio: porcentaje y si tiene parkinson o no.
                                         \\\hline
  Postcondiciones                        & \multicolumn{2}{p{10.25cm}}{Se muestra el texto del resultado de la predicción.} \\\hline
  Excepciones                        & \multicolumn{2}{p{10.25cm}}{Ninguna.}\\\hline
  Importancia                            & Alta \\\hline
  Urgencia                               & Alta \\
}

%CU6
\tablaSmallSinColores{Caso de uso 6: Elegir audio de la lista}{p{3cm} p{.75cm} p{9.5cm}}{tablaUC6}{
  \multicolumn{3}{l}{Caso de uso 6: Elegir audio de la lista} \\
 }
 {
  Descripción                            & \multicolumn{2}{p{10.25cm}}{Permite al usuario elegir un audio de los cargados para su procesamiento.} \\\hline
  \multirow{2}{3.5cm}{Requisitos}   &\multicolumn{2}{p{10.25cm}}{RF-1.1} \\\cline{2-3}
  									& \multicolumn{2}{p{10.25cm}}{RF-2.1} \\\cline{2-3}
  									& \multicolumn{2}{p{10.25cm}}{RF-3.1} \\\cline{2-3}
  									& \multicolumn{2}{p{10.25cm}}{RF-4.2} 
                                         \\\hline
  Precondiciones                         &  \multicolumn{2}{p{10.25cm}}{Tener un audio cargado}  \\\hline
  \multirow{2}{3.5cm}{Secuencia normal}  & Paso & Acción \\\cline{2-3}
                                         & 1    & El usuario elige pinchando un audio de la lista.
  \\\cline{2-3}
                                         & 2    & El audio se remarca en azul.
                                         \\\hline
  Postcondiciones                        & \multicolumn{2}{p{10.25cm}}{Es posible la reproducción o predicción del audio seleccionado.} \\\hline
  Excepciones                        & \multicolumn{2}{p{10.25cm}}{Audio no seleccionado.}\\\hline
  Importancia                            & Alta \\\hline
  Urgencia                               & Alta \\
}

%CU7
\tablaSmallSinColores{Caso de uso 7: Crear modelos}{p{3cm} p{.75cm} p{9.5cm}}{tablaUC7}{
  \multicolumn{3}{l}{Caso de uso 7: Crear modelos} \\
 }
 {
  Descripción                            & \multicolumn{2}{p{10.25cm}}{El científico de datos crea los modelos de predicción.} \\\hline
  \multirow{2}{3.5cm}{Requisitos}   &\multicolumn{2}{p{10.25cm}}{RF-5} \\\cline{2-3}
                                         & \multicolumn{2}{p{10.25cm}}{RF-5.1} \\\cline{2-3}
                                         & \multicolumn{2}{p{10.25cm}}{RF-5.2} \\\cline{2-3}
                                         & \multicolumn{2}{p{10.25cm}}{RF-5.3} \\\cline{2-3}
                                         & \multicolumn{2}{p{10.25cm}}{RF-5.4} \\\cline{2-3}
                                         & \multicolumn{2}{p{10.25cm}}{RF-5.5}  
                                         \\\hline
  Precondiciones                         &  \multicolumn{2}{p{10.25cm}}{Tener las características de los audios preextraídas}  \\\hline
  \multirow{2}{3.5cm}{Secuencia normal}  & Paso & Acción \\\cline{2-3}
                                         & 1    & El científico de datos elige un modelo de python a entrenar.
  \\\cline{2-3}
                                         & 2    & El científico de datos elige los datos con los que entrenar.
    \\\cline{2-3}
                                         & 3    & El científico de datos elige el tipo de validación cruzada. 
    \\\cline{2-3}
                                         & 4    & El científico de datos elige como particionar los datos.
     \\\cline{2-3}
                                         & 5    & El modelo se entrena.
                                         \\\hline
  Postcondiciones                        & \multicolumn{2}{p{10.25cm}}{Tenemos un modelo ya entrenado.} \\\hline
  Excepciones                        & \multicolumn{2}{p{10.25cm}}{Ninguna}\\\hline
  Importancia                            & Alta \\\hline
  Urgencia                               & Alta \\
}


%CU8
\tablaSmallSinColores{Caso de uso 8: Evaluar modelos}{p{3cm} p{.75cm} p{9.5cm}}{tablaUC8}{
  \multicolumn{3}{l}{Caso de uso 8: Evaluar modelos} \\
 }
 {
  Descripción                            & \multicolumn{2}{p{10.25cm}}{Permite al científico de datos realizar una comparativa de resultados de los modelos.} \\\hline
  \multirow{2}{3.5cm}{Requisitos}   &\multicolumn{2}{p{10.25cm}}{RF-6} \\\cline{2-3}
  									& \multicolumn{2}{p{10.25cm}}{RF-6.1} \\\cline{2-3}
  									& \multicolumn{2}{p{10.25cm}}{RF-6.2} \\\cline{2-3}
  									& \multicolumn{2}{p{10.25cm}}{RF-6.3} 
                                         \\\hline
  Precondiciones                         &  \multicolumn{2}{p{10.25cm}}{Haber entrenado los modelos}  \\\hline
  \multirow{2}{3.5cm}{Secuencia normal}  & Paso & Acción \\\cline{2-3}
                                         & 1    & El científico de datos extrae las medidas de rendimiento de varios modelos.
  \\\cline{2-3}
                                         & 2    & El científico de resume las medidas de rendimiento en gráficas o tablas.
    \\\cline{2-3}
                                         & 3    & El científico de datos visualiza las medidas de rendimiento en gráficas o tablas. 
                                         \\\hline
  Postcondiciones                        & \multicolumn{2}{p{10.25cm}}{Vemos los resultados de la evaluación de los modelos.} \\\hline
  Excepciones                        & \multicolumn{2}{p{10.25cm}}{Ninguna.}\\\hline
  Importancia                            & Alta \\\hline
  Urgencia                               & Alta \\
}





